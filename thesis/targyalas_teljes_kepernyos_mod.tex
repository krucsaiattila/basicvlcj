\section{Teljes képernyős mód}

Az eddig elkészült alkalmazás már képes videofájlok lejátszására, illetve alapvető vezérlési funkciókat is ismer, azonban a szoftver egy fontos része hiányos, tekintve, hogy elsősorban film lejátszására ajánlott. Ez pedig a teljes képernyős mód. Az alkalmazás, amint a felhasználó megnyomja a teljes képernyős mód gombot, kitölti a képernyőfelület egészét, eltüntetve ezzel a menüsávot és a kezelőpanelt.

Az új funkció alkalmazása azonban egy új problémát vetett fel: ha a képernyő egésze foglalt, és nem jelenik meg a kezelősáv, akkor a felhasználó nem tudja a teljes képernyős módot bezárni. Ennek megoldására egy általánosan használt módszert alkalmaztam. Ha a felhasználó a kurzort a képernyő alsó felére húzza, a vezérlőpanel megjelenik, és megjelenítve is marad mindaddig, amíg a felhasználó el nem távolítja az alsó területről a kurzort. Itt egyszerűen a teljes képernyős mód gombjára kattintva kiléphetünk abból. Ennek a megvalósításáért, illetve a panel megjelenítésének kezeléséért egy \textit{MouseListener} felel. Amennyiben az alkalmazás teljes képernyős módban van, a figyelő ellenőrzi a kurzor aktuális Y koordináta szerinti pozícióját, és ha ez egy bizonyos szám alatti, vagy feletti, akkor a szerint jeleníti meg, valamint rejti el a vezérlőpanelt, illetve ennek megfelelően igazítja a vásznat és a feliratokat (\ref{lst:mouselistener}).

\begin{lstlisting}[caption=A \textit{MouseListener} megvalósítása, label={lst:mouselistener}, language=java]
@Override
public void mouseMoved(MouseEvent e) {
   if(mediaPlayer.isFullScreen()) {
        if (e.getYOnScreen() < Toolkit
            .getDefaultToolkit()
            .getScreenSize().height-5
        && controlsPanel.isVisible()) {
            controlsPanel.setVisible(false);
            ((SubtitleOverlay)(mediaPlayer.getOverlay()))
            .decreaseYOffset(controlsPanel.getHeight());
        } else if (e.getYOnScreen() >= Toolkit
            .getDefaultToolkit()
            .getScreenSize().height-5
        && !controlsPanel.isVisible()) {
            controlsPanel.setVisible(true);
            ((SubtitleOverlay(mediaPlayer.getOverlay()))
            .increaseYOffset(controlsPanel.getHeight());
        }
    }
}
\end{lstlisting}

Ezen felül további kényelmi funkciókat is implementáltam a hatékonyabb teljes képernyős mód kezeléséért. A mód aktiválható a vászonra történő dupla kattintással, és meg is szüntethető ugyanezzel a módszerrel. Ehhez szintén egy \textit{MouseListener}-t hívtam segítségül, amely figyeli az egymás utáni kattintások számát, és ha ez az érték kettővel egyenlő, be- valamint kilép a teljes képernyős módból. Egy másik megvalósított segédfunkció az \textit{Esc} gombbal történő vezérlés. Amennyiben az alkalmazás teljes képernyőn fut, a mód megszakítható az \textit{Esc} gomb egyszeri megnyomásával. Mindkét lehetőség kényelmesebb, gyorsabb használatot tesz lehetővé, ezzel is növelve a felhasználói komfortérzetet.

