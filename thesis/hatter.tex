\chapter{Háttér}
\addcontentsline{toc}{section}{Háttér}
\spacing{1.5}
Az alkalmazás elkészítése előtt kutatást végeztem a témában lehetséges alternatívaként felmerülő szoftverekről. Tehát olyan megoldásokat kerestem, amelyek segítségével a felhasználó képes feliratok manipulálására, nyelvi fordítására, amely segítségével tudását bővítheti, elmélyítheti. Ezen kutató, illetve háttérmunkát rendkívül fontosnak tartom, hiszen ezáltal betekintést nyerhetünk a piaci igényekbe, következtetéseket vonhatunk le a már elérhető programok palettájából. A kutatás során elsődleges forrásként az internetet használtam, tehát a jelen dolgozatban bemutatott opciók nagy részben weboldalakról származnak. Itt rengeteg hasonló célokat szolgáló szoftver, honlap található, azonban ezek funkcionalitása eltér az általam fejleszteni kívánt alkalmazástól. Ezek mellett fontosnak tartom megemlíteni azt a tényt is, hogy az előzőekben említett, általam elkészített alkalmazás ingyenesen elérhető lesz, szemben a világhálón található hasonló alkalmazásokkal.

Kutatásom első részében olyan programokat kínáló megoldásokat kerestem, melyek képesek a feliratfájlok szövegét felismerni, valamint lefordítani. Erre számtalan online alkalmazás, illetve felület képes. Ezek a megoldások többnyire ingyenesek, mivel a weblapok rendszerint nagyméretű reklámfelületeket tartalmaznak. A működésük egyszerű: a felhasználó vagy drag\&drop módszerrel, vagy tallózással kiválasztja az általa lefordítani kívánt megfelelő kiterjesztésű fájlt, amelyet feltölt a weboldalra. Ezután a képernyőn pillanatok alatt megjelenik az eredeti, valamint a lefordított szöveg. A felhasználó itt kiválaszthatja, módosíthatja a fordítás nyelvét, mind a fordítandó, mind pedig a fordított nyelvet. Legtöbbször nyelvfelismerő szolgáltatás is működik, amely nagy biztonsággal képes meghatározni a forrásfájl nyelvét, ezzel is gyorsítva a fordítás folyamatát. Ha a felismerő hibázna lehetőség van manuálisan is megadni a nyelvet. Ezen megoldások nagy előnye, hogy gyorsak, könnyedén elérhetőek, illetve nem szükséges semmiféle külön alkalmazás telepítése a számítógépre. Működésükhöz csupán egy böngésző, valamint internetkapcsolat szükségeltetik, így ezek akár mobileszközökön is használhatók. E megoldás nagy hátránya a pontatlanság, mivel a szöveget online fordítók segítségével alakítja át az alkalmazás. Az így keletkezett mondatok legtöbbször helytelenek, sok bennük a nyelvtani hiba, legtöbbször értelmetlenek. További hátrányuk, hogy működésükhöz elengedhetetlen az internetkapcsolat. Mivel fordításhoz feliratfájl feltöltése szükséges, így ebből adódóan gyenge internetelérés esetén a sebességük drasztikusan lecsökkenhet, még akkor is, ha maguk a fájlok méretben nem túl nagyok.

Egy másik opcióként kínálkoznak a fentebb említett megoldáshoz nagyon hasonló szoftverek. Működésük majdnem analóg az online szövegfordítókkal, azonban lényeges különbség, hogy a fordítást videolejátszás közben is képes végrehajtani. Online ingyenesen elérhetőek, telepítést követően teljes értékű videolejátszó alkalmazásként képesek funkcionálni. Megfelelő médiafájl kiválasztása után lehetőség van feliratok fájlból történő megjelenítésére is. Az alkalmazás online fordítók segítségével alakítja át az eredeti szöveget a célnyelvre. Működésük gyors, hatékony, azonban felhasználói felületük sokszor hiányos, nem felel meg a mai kor elvárásainak. Továbbá, mivel a felirat sorait egyben kezelik, így a fordítások legtöbbször pontatlanok, hiányosak, valamint nyelvtanilag helytelenek. Fő előnyük tehát, hogy filmnézés közben is képes lefordított feliratokat megjeleníteni úgy, hogy nem szükséges a feliratokat előzőlegesen lefordítani. Egyes fajták arra is lehetőséget biztosítanak a felhasználónak, hogy a filmhez saját fordításokat készítsen. Mivel a használatukhoz az alkalmazást kötelező telepíteni a számítógépre, így működéséhez csak akkor szükséges internetkapcsolat, amikor a fordítás történik. Ekkor kevesebb az adatforgalom, mint az előzőleg említett megoldásnál, mivel nem szükséges fájlok feltöltése egy külső szerverre. Hátrányként megemlíthető az a tény, hogy a szoftverek a feliratokat csak egészben, legfeljebb soronként tudják lefordítani, így a keletkező szöveg rendkívül pontatlan.

Harmadik lehetőségként egy speciális alternatívát ismertetnék. Ezen belül olyan weboldalakat, szoftvereket kell megemlítsek, ahol a fordítás nem gépi módon, automatikusan, esetleg mesterséges intelligencia segítségével történik, hanem valós emberek, személyek által. Egy weboldalt kiemelnék,  és segítségével ismertetném a fent említett fordítási módszer működését. Az oldal neve \textit{Amara}, amely weboldalán megtalálható a részletes leírás, ismertető, amely bemutatja a felhasználónak az oldal használatát. Alapvetően kétféle felhasználó jogosultsági szintet különböztethetünk meg. Az első az ingyenes, ahol a felhasználó csak az oldal alapfunkcióit éri el. Ez kimerül feliratok készítésében, videofájlokra illesztésében, illetve a létrejött feliratok manuális lefordításában. Ezek egy online felületen keresztül tehetők meg. Előnye, hogy gyors, kényelmes, felhasználóbarát, kiválthatóak vele a komplikált videoszerkesztő szoftverek bizonyos funkciói. Hátrányai közt említhető, hogy csak online érhető el, így folyamatos internetkapcsolatra van szükség, illetve maga a fordítási funkció teljes mértékben a felhasználó tudásán alapul, neki semmiféle nyelvi támogatást nem nyújt. A másik felhasználói szint díjköteles. Ezt igénybe véve a felhasználó azon kívül, hogy természetesen használhatja az ingyenes funkciókat, rengeteg új opció is nyílik előtte. Ezek közül a legfontosabb az, ahol videofájlok feltöltése után különböző anyanyelvű segítők készítenek feliratokat az általunk feltöltött videóhoz. Ez azonban idő- és erőforrás igényes, használata -az oldal szerint- kifejezetten vállalatoknak ajánlott. Ezen módszer hatalmas előnye, hogy a készített felirat nyelvtanilag és szemantikailag biztosan helyes, gépi fordítási hibáktól mentes. Gyenge pontja azonban az, hogy otthoni, felhasználói használatra kevésbé alkalmas, mint a fentebb említett opciók, mivel a felirat precíz elkészítése az emberi erőforrás bevonása miatt időigényes, valamint költséges. Továbbá kevésbé hatékony nyelvtanulásra, hiszen a felhasználónak a lefordított felirattal semmiféle interakciója nem lehet, az készen kerül elé. Emiatt, ha filmnézésre szeretnénk használni, ezt tehetjük egyszerűen anyanyelvű felirattal, ami kevésbé ösztönzi a felhasználót tanulásra, gondolkodásra, problémamegoldásra.

Összességében megállapítható, hogy olyan, kimondottan nyelvtanulást segítő alkalmazás, amelynek funkciói az általam elkészített programoméhoz hasonlóak, nem létezik, illetve kutatásom során nem találkoztam eggyel sem.