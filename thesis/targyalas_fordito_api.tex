\section{Feliratok fordítása}

Miután már lehetőségünk van arra, hogy a felhasználó szavakra történő kattintásait nyomon kövessük, a következő lépés ezek lefordítása a kívánt nyelvre. Ehhez szükséges egy fordító \textit{API}-ra, azaz egy alkalmazásprogramozási felületre, amely segítségével megvalósíthatóvá válik a szavak anyanyelvre való átültetése. Ehhez a \textit{Microsoft Translate API} interfészre esett a választásom, mivel az összes alternatíva közül ez az egyetlen ingyenes. Az \textit{API} működésének lényege, hogy a videólejátszó alkalmazás interneten keresztül elküldi a megadott címre a fordítani kívánt szót, majd innen egy \textit{JSON} formátumú válasz érkezik, ami tartalmazza az összes információt, -beleértve a fordításokat is- amelyet a szerver képes volt előállítani az elküldött információk alapján.
A szolgáltatás többféle fordítási módot is támogat:
\begin{itemize}[noitemsep]
\item fordítás
\item transzliteráció
\item szófordítás
\end{itemize}
Az alkalmazás szempontjából az utóbbi lesz megfelelő, mivel ez a funkció lehetővé teszi egy szóhoz tartozó több fordítás megjelenítését is, amely a felhasználó szókincsének bővítésére rendkívül hasznos.

\subsection{Szerverrel történő kommunikáció}
A szolgáltatás használatához regisztrálást követően igényelnem kellett egy egyedi kulcsot, amely a fiókom azonosításra szolgál. A \textit{TranslateService.java} osztály valósítja meg az információáramlást szerver és kliens között. Mivel az \textit{API} egy másik URL-en képes nyelvfelismerésre is, ezért az osztályba két publikus metódus került. Egy, amely a kiválasztott nyelv esetén hívódik, egy másik pedig amikor a nyelvfelismerés aktív. 

Kiválasztott nyelv esetén alkalmazott függvény:
\begin{verbatim}
public TranslateResponse[] PostWithGivenLanguages(
    String from,
    String to,
    String string) throws IOException {
        givenLanguageUrl += from + "&to=";
        givenLanguageUrl += to;

        Gson gson = new GsonBuilder()
            .setPrettyPrinting()
            .create();
        return gson.fromJson(createResoponse(string,
            givenLanguageUrl).
        body().string(), TranslateResponse[].class);
    }
\end{verbatim}
Ebben az esetben a metódus három paraméterrel rendelkezik:
\begin{itemize}[noitemsep]
\item String from, a forrásnyelv
\item String to, a célnyelv
\item String string, a fordítandó szöveg
\end{itemize}
Működése közben előállít egy URL-t, amely tartalmazza a fent említett forrás- és célnyelvet, valamint a felhasználáshoz szükséges kulcsot, beégetett módon. A \textit{createResponse} metódus felelős a szerverrel történő kommunikációért, beleértve a kérés elküldését, illetve a válasz átadását is. A kapott \textit{JSON} formátumú szöveget \textit{Gson} segítségével deszerializálja, így jön létre egy \textit{TranslateResponse} tömb, amely a szervertől kapott választ hivatott reprezentálni. Ez tartalmazza az elérhető fordításokból alkotott listát, mellyel tovább dolgozhatunk.

Nyelvdetektálás esetén alkalmazott függvény:
\begin{verbatim}
public DetectedLanguageResponse[] PostWithLanguageDetection(
    String string) throws IOException {
        Gson gson = new GsonBuilder()
            .setPrettyPrinting()
            .create();
        return gson.fromJson(createResoponse
            (string,
            "https://api.cognitive.microsofttranslator.com
            /detect?api-version=3.0")
                .body().string(),
            DetectedLanguageResponse[].class);
    }
\end{verbatim}
Az előző metódushoz képest annyi különbség fedezhető fel, hogy a paraméterlistában csak a lefordítandó szöveg szerepel, és a POST kérés másik címre fut be. Ezen kívül a feldolgozott \textit{JSON} egy \textit{DetectedTranslateResponse}-okat tartalmazó tömbbe kerül. 

\subsection{Fordítások megjelenítése}
