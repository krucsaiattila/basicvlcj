\section{Architekturális áttekintés}

Kutatás után az alkalmazás megvalósításához szükséges technológiákat gyűjtöttem össze. Mivel ezt egy asztali, telepíthető szoftvernek szántam, így a \textit{Java} nyelv, azon belül is a \textit{Java 8} mellett döntöttem. A választás azért előnyös, mivel a nyelv platformfüggetlen, így különböző operációs rendszerekre is telepíthető lesz az alkalmazás, valamint nagy mennyiségű API-k állnak rendelkezésre. A projekt, illetve a függőségek kezelésére \textit{maven 3.6.0}-as verzióját használtam. A külső könyvtárak kezelése ezzel  gyorsabbá, egyszerűbbé válik. A szoftver videolejátszó funkcióit a \textit{vlcj} keretrendszer segítségével valósítottam meg, amely a népszerű \textit{VLC} videolejátszó alkalmazáshoz egy natív hívásokkal operáló \textit{API}, amely segítségével lehetőség nyílik a \textit{VLC} funkcionalitásait \textit{Java} nyelven használni. A program grafikai felülete \textit{Java Swing} segítségével íródott, s mivel a \textit{vlcj} is támogatja az ebbe történő beágyazást, így egyszerű volt azt kiegészíteni, továbbfejleszteni. Az adatok tárolására szükségem volt egy adatbázis kezelő rendszerre. Mivel a a szükséges adatbázis struktúra egyszerű, illetve az elvégzendő műveletek tárháza csekély, ezért végül a \textit{H2} adatbázist választottam. Ennek nagy előnye, hogy képes \textit{in-memory} adatbázis kezelésére, valamint a használatához nem szükséges további csomagok feltelepítése a számítógépre, egyszerűen a megfelelő \textit{.jar} fájl importálásával elérhetővé válik az adatbázis. A program egyik legfontosabb funkciója, a szavak, mondatelemek fordítása a \textit{Microsoft Translate API} beiktatásával valósult meg. Az API képes több, mint 60 nyelvfelismerésére, illetve fordítására. Használata egyszerű, működése gyors, pontos, illetve ami még fontos volt a projekt szempontjából, hogy képes egynél több találat megjelenítésére is.

A fejlesztés \textit{Windows 10} operációs rendszer alatt történt. A programozáshoz, futtatáshoz, teszteléshez \textit{IntelliJ IDEA} integrált fejlesztői környezetet használtam. A szoftver verziókövetése, illetve a fájlok tárolása \textit{GitHub} repository-n keresztül valósult meg.