\chapter*{Feladatkiírás}
%A tartalomjegyzékben mégis szerepeltetni kell, mint szakasz(section) szerepeljen:
\addcontentsline{toc}{section}{Feladatkiírás}
\spacing{1.5}


Az idegennyelv tanulás folyamán sokszor nemcsak a tankönyvekből tanulunk. Az internet, könyvek, újságok, filmek mind szerves részét képezhetik egy idegennyelv elsajátításának. A szoftver azoknak az egyéneknek nyújthat segítséget, akik rendszeresen néznek idegennyelvű filmeket, idegennyelvű felirattal. A felhasználó egy ismeretlen szó esetén megállíthatja a lejátszást, és a mondatelemre kattintva megtudhatja az anyanyelvi jelentését. Így a szótározással eltöltött idő lényegesen kevesebb, kevésbé szakad meg a film folyamatossága.
Az ismeretlen szavakat az alkalmazás egy adatbázisban tárolja, amelyből a film befejeztével egy szószedet generálódik a filmből kiragadott példamondatokkal. A létrejött fájlt el lehet menteni, így a felhasználó a későbbiekben is átnézheti a nehezebben megjegyezhető szavakat.
Egy másik lehetőség a film véget értével, hogy az alkalmazás egy feleletválasztós kvízt generál, amely a korábban ismeretlen mondatelemeket tartalmazza. A tanuló így ellenőrizheti a saját tudását, valamint azt, hogy mennyire sikerült elsajátítania az idegen szavak jelentéseit.

