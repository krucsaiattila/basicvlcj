\chapter*{Bevezetés}
\addcontentsline{toc}{section}{Bevezetés}
\onehalfspace
Szakdolgozatom célja, mint ahogy az már az előző bekezdésben ismertetésre került, egy a nyelvtanulást elősegítő videolejátszó alkalmazás fejlesztése, melynek segítségével a felhasználók a szórakozást tanulással köthetik össze. Ennek eredményeképp úgy tudnak elsősorban idegennyelvű szavakat elsajátítani, hogy mellőzik a tanulással járó feszültséget, illetve kötöttséget. Tehát a nyelvek szókincsének elsajátítása, elmélyítése, tudatosítása, lehető legbiztonságosabb használata ezáltal könnyedebbé, szórakoztatóbbá és gördülékenyebbé válhat, mely egy hatékonyabb tanulási folyamatot tesz lehetővé. Továbbá feltételezhető, hogy az általam elkészített alkalmazás elsősorban filmek lejátszására ajánlott, így a felhasználok egy sokkal inkább a hétköznapi kommunikációhoz alkalmazható szókincsre tehetnek szert. Ezen szoftver segítségével az idegennyelv-tanulástól ódzkodó személyek érdeklődése nagymértékben felkelthető, mivel a tanulás nem direkt módon, lexikálisan történik, hanem filmnézés, szórakozás közben a szavak az idegennyelven hallott és látott szövegben, kontextusban rögzülnek. 

Napjainkban egyre inkább felmerül a kérdés, hogy a nyelvtanulást milyen módszerekkel lehet hatékonyabbá, illetve játékosabbá tenni. Kétségtelen azonban, hogy egy idegen nyelv magabiztos szintű elsajátításához a megfelelő méretű szókincs birtoklása elengedhetetlen. A szavak memorizálása sokszor monoton folyamat, valamint rengeteg ráfordított időt igényel. Ennek kapcsán saját tapasztalatból is biztonsággal állíthatom, hogy az idegen nyelvű szavak elsajátítása megítélésem szerint a jelenleg használatos oktatási módszerekkel igen monoton, mely a diákok motiválatlanságát, elidegenülését eredményezheti. Mivel a motiváció hiánya gyakran tanulási nehézségeket eredményez, a nagy erőfeszítések árán megtanult szavak nem rögzülnek megfelelően, ha igen, akkor sokszor hibásan, így gyorsan el is felejtődnek. A jelenlegi, fejlett informatikai világban a fiatal generáció mindennapi életében természetes és elengedhetetlen az informatikai eszközök használata, ami lehetővé teszi az idegennyelv tanulás elősegítését is. Az informatikai eszközök sokoldalúsága, hordozhatósága, a nap minden órájában lehetővé teszi a használatot, amely a filmnézéssel és szórakozással egybekötött nyelvtanulást is elősegíti, így fiatal generáció életvitelébe teljes mértékben beilleszthető.
 Egy alternatívaként merülhet fel a tanulásra az a videolejátszó alkalmazás, amelyet szakdolgozatom keretein belül készítek el, azt a célt kitűzve magam elé, hogy közelebb hozzam egymáshoz a tanulást és a szórakozást. 

Egy idegennyelv elsajátítása során sokszor nemcsak tankönyveket használunk. Az internet, könyvek, újságok, filmek mind szerves részét képezhetik a tanulásnak. A szoftver azoknak az egyéneknek nyújthat nagymértékben segítséget, akik rendszeresen néznek idegennyelvű filmeket, idegennyelvű felirattal. A felhasználó egy a számára ismeretlen szó esetén a film megállítása nélkül a mondatelemre kattintva megtudhatja az anyanyelvi jelentését. Így a szótározással eltöltött idő megszűnhet, mely eredményeképp a megtekintett film folyamatossága biztosított. Azon szavakat, melyek anyanyelvi megfelelőjét a felhasználó filmezés közben megtekinti, az alkalmazás egy adatbázisban tárolja, amelyből a film befejeztével egy szószedet generálódik. Az így összeállt lista különlegessége, hogy nem csak az egyes szavak idegen- illetve anyanyelvű alakjait tartalmazza, hanem ezeket a filmből kiragadott példamondatokkal abban a kontextusban szemlélteti, amelyben az adott szó korábban előfordult. A létrejött fájl automatikusan mentésre kerül a felhasználó eszközére, így a későbbiekben is megtekinthetővé válik a lista. Egy további opcióként kínálkozik az a lehetőség, hogy a film végeztével az alkalmazás egy feleletválasztós kvízt generál azokból a mondatelemekből, melyeket a felhasználó korábban ismeretlenként jelölt meg. Ezáltal lehetősége nyílik a megszerzett tudás ellenőrzésére is.

Alkalmazásom célja tehát egy olyan alternatív lehetőség felkínálása az idegennyelvet tanulók részére, amellyel az ismeretlen szavak memorizálása szórakozással köthető össze, amely által az idegennyelv elsajátításával járó stressz csökkenthető, esetlegesen elkerülhetővé válik, így nagy eséllyel sikerélmény keletkezik, ezáltal tanulás hatékonysága is nő.
