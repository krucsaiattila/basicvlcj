\chapter*{Bevezetés}
\addcontentsline{toc}{section}{Bevezetés}

Diplomamunkám során egy nyelvtanulást elősegítő videolejátszó  alkalmazás fejlesztése a célom, amellyel könnyedebben, illetve hatékonyabban lehet elsajátítani az idegen szavakat, mindezt szórakozással egybekötve.

Problémafelvetés: Egy idegen nyelv magabiztos szintű elsajátítása megfelelő méretű szókincset igényel. A szavak memorizálása sokszor monoton folyamat, valamint rengeteg ráfordított időt igényel. Egy alternatívaként merülhet fel a tanulásra az a videolejátszó alkalmazás, amelyet diplomamunkám keretein belül készítek el, mely közelebb hozza egymáshoz a tanulást és a szórakozást.

Az idegennyelv tanulás folyamán sokszor nemcsak a tankönyvekből tanulunk. Az internet, könyvek, újságok, filmek mind szerves részét képezhetik egy idegennyelv elsajátításának. A szoftver azoknak az egyéneknek nyújthat segítséget, akik rendszeresen néznek idegennyelvű filmeket, idegennyelvű felirattal. A felhasználó egy ismeretlen szó esetén a film megállítása nélkül a mondatelemre kattintva megtudhatja az anyanyelvi jelentését. Így a szótározással eltöltött idő lényegesen kevesebb, kevésbé szakad meg a film folyamatossága.
Az ismeretlen szavakat az alkalmazás egy adatbázisban tárolja, amelyből a film befejeztével egy szószedet generálódik a filmből kiragadott példamondatokkal. A létrejött fájlt el lehet menteni, így a felhasználó a későbbiekben is átnézheti a nehezebben megjegyezhető szavakat.
Egy másik lehetőség a film véget értével, hogy az alkalmazás egy feleletválasztós kvízt generál, amely a korábban ismeretlen mondatelemeket tartalmazza. A tanuló így ellenőrizheti a saját tudását, valamint azt, hogy mennyire sikerült elsajátítania az idegen szavak jelentéseit.

Az alkalmazás célja tehát az, hogy egy olyan alternatív lehetőséget teremtsen az idegennyelven tanulók számára, amellyel az ismeretlen szavak memorizálása szórakozással és filmnézéssel köthető egybe, így növelve a tanulás hatékonyságát.

