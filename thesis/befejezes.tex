\chapter*{Befejezés}

Szakdolgozatom célja egy a nyelvtanulást elősegítő alkalmazás fejlesztése volt, amely segítségével a felhasználók szórakozással egybekötve, filmnézés közben tehetnek szert értékes nyelvtudásra. Ez hatványozottan érvényes szavak memorizálása esetén, hiszen a filmekhez készült idegennyelvű feliratok kitűnő forrást biztosítanak bármely nyelv elsajátításához. A tanulás folyamata így könnyedebbé, szórakoztatóbbá válik, illetve a feliratok szóhasználatából eredendően, leginkább a hétköznapi szókincs fejlesztésére alkalmasabb. Mindezek mellett a szavak memorizálására fordított idő lecsökkenhet, valamint ennek folyamata hatékonyabbá válhat. 

A megvalósított funkciók mind a fentebb említett nyelvtanulási módszerek színesítését szolgálják. A videólejátszó alapfunkciók kényelmes és felhasználóbarát kezelést tesznek lehetővé, egészen a lejátszás irányításától, a teljes képernyős funkción át, a média- és feliratfájlok kiválasztásának lehetőségéig. Emellett a szoftver képes feliratok dinamikus megjelenítésére, valamint a felhasználó szavakra történő interakcióit, egérkattintásait is képes kezelni. A néző több mint 60 nyelv közül választhat, amelyeken a szoftver képes fordításokat végezni. Így közel 3600 féle nyelvpár közül válogathat a felhasználó, amely a legtöbb esetben kielégít bármilyen igényt. Ha nem áll rendelkezésünkre feliratfájl, az online feliratkeresővel több ezer, interneten megtalálható fájl között böngészhetünk, mindezt kényelmesen, az alkalmazás elhagyása nélkül. A beállított felirat és a film, valamint felhasználói képernyőméret és felirat betűméret összhangjának megteremtése érdekében valósítottam meg a beállítások panelt, amellyel mindez egyszerűen létrehozható. Tudásunk ellenőrzésére két alternatíva közül is választhatunk. Első lehetőségként a generált szószedet kínálkozik, mely segítségével a tanuló átismételheti az általa ismeretlennek vélt szavakat. Ez tartalmazza az idegen szót, jelentéseit, illetve a kontextust, melyben az adott mondatelem szerepelt. Emellett lehetőségünk van egy tudásellenőrző kvíz kitöltésére is, amely az idegen szavak memorizálásához nyújt segítséget játékos módon.

Jövőbeli tervek között szerepel, további telepítőfájlok elkészítése különböző operációs rendszerekhez. Hasznos lenne egy új funkció is, amely a kattintott szót felolvassa, kiejti, így még több információhoz juttatva a felhasználót. Valamint a szóhoz egy teljes értékű szócikket is megjelenítene, hasonlót mint az egynyelvű szótárakban szereplő. Ebből a tanuló megtudhatná a szó szófaját, definícióját, fonetikai átírását, szinonímáit, morfológiai tulajdonságai, valamint példamondatokat is láthatna egyes jelentéseire. Mindezek mellett a szoftver jövőbeli teljes körű támogatását tervezem.