\documentclass[12pt]{report}

\usepackage[utf8]{inputenc}
\usepackage[T1]{fontenc}
\usepackage[magyar]{babel}

\usepackage{times}

\usepackage{amsmath}
\usepackage{amssymb}
\usepackage{amsthm}

\usepackage{fancyhdr}

\usepackage{graphicx}
\usepackage{psfrag}

\usepackage{url}

\newtheorem{tét}{Tétel}[chapter]
\newtheorem{defi}[tét]{Definíció}
\newtheorem{lemma}[tét]{Lemma}
\newtheorem{áll}[tét]{Állítás}
\newtheorem{köv}[tét]{Következmény}

\theoremstyle{definition}
\newtheorem{megj}[tét]{Megjegyzés}
\newtheorem{pld}[tét]{Példa}

%Margók:
\hoffset -1in
\voffset -1in
\oddsidemargin 35mm
\textwidth 150mm
\topmargin 15mm
\headheight 10mm
\headsep 5mm
\textheight 237mm

\begin{document}

%A FEJEZETEK KEZDÕOLDALAINAK FEJ ES LÁBLÉCE:
%a plain oldalstílust kell átdefiniálni, hogy ott ne legyen fejléc:
\fancypagestyle{plain}{%
%ez mindent töröl:
\fancyhf{}
% a láblécbe jobboldalra kerüljön az oldalszám:
\fancyfoot[R]{\thepage}
%elválasztó vonal sem kell:
\renewcommand{\headrulewidth}{0pt}
}

%A TÖBBI OLDAL FEJ ÉS LÁBLÉCE:
\pagestyle{fancy}
\fancyhf{}
\fancyhead[L]{A diplomamunka címe}
\fancyfoot[R]{\thepage}


%A címoldalra se fej- se lábléc nem kell:
\thispagestyle{empty}

\begin{center}
\vspace*{1cm}
{\Large\bf Szegedi Tudományegyetem}

\vspace{0.5cm}

{\Large\bf Informatikai Intézet}

\vspace*{3.8cm}


{\LARGE\bf A diplomamunka címe}


\vspace*{3.6cm}

{\Large Diplomamunka}
% vagy {\Large Szakdolgozat}

\vspace*{4cm}

%Értelemszerûen megváltoztatandó:
{\large
\begin{tabular}{c@{\hspace{4cm}}c}
\emph{Készítette:}     &\emph{Témavezetõ:}\\
\bf{Krucsai Attila}  &\bf{Tóth Zotán Gábor}\\
informatika szakos     &egyetemi tanársegéd\\
hallgató&
\end{tabular}
}

\vspace*{2.3cm}

{\Large
Szeged
\\
\vspace{2mm}
2019
}
\end{center}


%A tartalomjegyzék:
\tableofcontents


\chapter*{Feladatkiírás}
%A tartalomjegyzékben mégis szerepeltetni kell, mint szakasz(section) szerepeljen:
\addcontentsline{toc}{section}{Feladatkiírás}
\spacing{1.5}


Az idegennyelv tanulás folyamán sokszor nemcsak a tankönyvekből tanulunk. Az internet, könyvek, újságok, filmek mind szerves részét képezhetik egy idegennyelv elsajátításának. A szoftver azoknak az egyéneknek nyújthat segítséget, akik rendszeresen néznek idegennyelvű filmeket, idegennyelvű felirattal. A felhasználó egy ismeretlen szó esetén megállíthatja a lejátszást, és a mondatelemre kattintva megtudhatja az anyanyelvi jelentését. Így a szótározással eltöltött idő lényegesen kevesebb, kevésbé szakad meg a film folyamatossága.
Az ismeretlen szavakat az alkalmazás egy adatbázisban tárolja, amelyből a film befejeztével egy szószedet generálódik a filmből kiragadott példamondatokkal. A létrejött fájlt el lehet menteni, így a felhasználó a későbbiekben is átnézheti a nehezebben megjegyezhető szavakat.
Egy másik lehetőség a film véget értével, hogy az alkalmazás egy feleletválasztós kvízt generál, amely a korábban ismeretlen mondatelemeket tartalmazza. A tanuló így ellenőrizheti a saját tudását, valamint azt, hogy mennyire sikerült elsajátítania az idegen szavak jelentéseit.


\chapter*{Tartalmi összefoglaló}
\addcontentsline{toc}{section}{Tartalmi összefoglaló}

A tartalmi összefoglalónak tartalmaznia kell (rövid, legfeljebb egy oldalas, összefüggõ megfogalmazásban)
a következõket: a téma megnevezése, a megadott feladat megfogalmazása - a feladatkiíráshoz viszonyítva-,
a megoldási mód, az alkalmazott eszközök, módszerek, az elért eredmények, kulcsszavak (4-6 darab).

Az összefoglaló nyelvének meg kell egyeznie a dolgozat nyelvével. Ha a dolgozat idegen nyelven készül,
magyar nyelvû tartalmi összefoglaló készítése is kötelezõ (külön lapon), melynek terjedelmét a TVSZ szabályozza.

\chapter*{Bevezetés}
\addcontentsline{toc}{section}{Bevezetés}
\spacing{1.5}
Szakdolgozatom célja, egy a nyelvtanulást elősegítő videolejátszó alkalmazás fejlesztése, melynek segítségével a felhasználók a szórakozást tanulással köthetik össze. Ennek eredményeképp úgy tudnak elsősorban idegennyelvű szavakat elsajátítani, hogy mellőzik a tanulással járó feszültséget, illetve kötöttséget. Tehát a nyelvek szókincsének elsajátítása, elmélyítése, tudatosítása, lehető legbiztonságosabb használata ezáltal könnyedebbé, szórakoztatóbbá és gördülékenyebbé válhat, mely egy hatékonyabb tanulási folyamatot tesz lehetővé. Mivel az általam elkészített alkalmazás elsősorban filmek lejátszására ajánlott, így a felhasználok egy sokkal inkább a hétköznapi kommunikációhoz alkalmazható szókincsre tehetnek szert. Ezen szoftver segítségével az idegennyelv-tanulástól ódzkodó személyek érdeklődése nagymértékben felkelthető, mivel a tanulás nem direkt módon, hanem filmnézés, szórakozás közben történik, a szavak az idegennyelven hallott és látott szövegben, kontextusban rögzülnek. 

Napjainkban egyre inkább felmerül a kérdés, hogy a nyelvtanulást milyen módszerekkel lehet hatékonyabbá, illetve játékosabbá tenni. Kétségtelen azonban, hogy egy idegen nyelv magabiztos szintű elsajátításához a megfelelő méretű szókincs birtoklása elengedhetetlen. A szavak memorizálása sokszor monoton folyamat, valamint rengeteg ráfordított időt igényel. Ennek kapcsán saját tapasztalatból is biztonsággal állíthatom, hogy az idegen nyelvű szavak elsajátítása megítélésem szerint a jelenleg használatos oktatási módszerekkel igen monoton, mely a diákok motiválatlanságát, elidegenülését eredményezheti. Mivel a motiváció hiánya gyakran tanulási nehézségeket eredményez, a nagy erőfeszítések árán megtanult szavak nem rögzülnek megfelelően, ha igen, akkor sokszor hibásan, így gyorsan el is felejtődnek. A jelenlegi, fejlett informatikai világban a fiatal generáció mindennapi életében természetes és elengedhetetlen az informatikai eszközök használata, ami lehetővé teszi az idegennyelv tanulás elősegítését is. Az informatikai eszközök sokoldalúsága, hordozhatósága, a nap minden órájában lehetővé teszi a használatot, amely a filmnézéssel és szórakozással egybekötött nyelvtanulást is elősegíti, így a fiatal generáció életvitelébe teljes mértékben beilleszthető.
 Egy alternatívaként merülhet fel a tanulásra az a videolejátszó alkalmazás, amelyet szakdolgozatom keretein belül készítettem el, azt a célt kitűzve magam elé, hogy közelebb hozzam egymáshoz a tanulást és a szórakozást. 

Egy idegennyelv elsajátítása során sokszor nemcsak tankönyveket használunk. Az internet, könyvek, újságok, filmek mind szerves részét képezhetik a tanulásnak. A szoftver azoknak az egyéneknek nyújthat nagymértékben segítséget, akik rendszeresen néznek idegennyelvű filmeket, idegennyelvű felirattal. A felhasználó egy a számára ismeretlen szó esetén a film megállítása nélkül a mondatelemre kattintva megtudhatja az anyanyelvi jelentését. Így a szótározással eltöltött idő megszűnhet, mely eredményeképp a megtekintett film folyamatossága biztosított. Azon szavakat, melyek anyanyelvi megfelelőjét a felhasználó filmezés közben megtekinti, az alkalmazás egy adatbázisban tárolja, amelyből a film befejeztével egy szószedet generálódik. Az így összeállt lista különlegessége, hogy nem csak az egyes szavak idegen- illetve anyanyelvű alakjait tartalmazza, hanem ezeket a filmből kiragadott példamondatokkal, abban a kontextusban szemlélteti, amelyben az adott szó korábban előfordult. A létrejött fájl automatikusan mentésre kerül a felhasználó eszközére, így a későbbiekben is megtekinthetővé válik a lista. Egy további opcióként kínálkozik az a lehetőség, hogy a film végeztével az alkalmazás egy feleletválasztós kvízt generál azokból a mondatelemekből, melyeket a felhasználó korábban ismeretlenként jelölt meg. Ezáltal lehetősége nyílik a megszerzett tudás ellenőrzésére is.

Alkalmazásom célja tehát egy olyan alternatív lehetőség felkínálása az idegennyelvet tanulók részére, amellyel az ismeretlen szavak memorizálása szórakozással köthető össze, amely által az idegennyelv elsajátításával járó stressz csökkenthető, esetlegesen elkerülhetővé válik, így nagy eséllyel sikerélmény keletkezik, ezáltal tanulás hatékonysága is nő.


\chapter{Függelék}

\section{A program forráskódja}
A függelékbe kerülhetnek a hosszú táblázatok, vagy mondjuk egy programlista:
% A verbatim kornyezet hasznalatanal ügyeljünk rá, hogy az editor a szóközöjket át ne írja tab karakterekre!
\begin{verbatim}
   while (ujkmodosito[i]<0)
   {
      if (ujkmodosito[i]+kegyenletes[i]<0)
      {
         j=i+1;
         while (j<14)
         if (kegyenletes[i]+ujkmodosito[j]>-1) break;
         else j++;
         temp=ujkmodosito[j];
         for (l=i;l<j;l++) ujkmodosito[l+1]=ujkmodosito[l];
         ujkmodosito[i]=temp;
      }
      i++;
   }
\end{verbatim}

\chapter*{Nyilatkozat}
%Egy üres sort adunk a tartalomjegyzékhez:
\addtocontents{toc}{\ }
\addcontentsline{toc}{section}{Nyilatkozat}
%\hspace{\parindent}

% A nyilatkozat szövege más titkos és nem titkos dolgozatok esetében.
% Csak az egyik tipusú myilatokzatnak kell a dolgozatban szerepelni
% A ponok helyére az adatok értelemszerûen behelyettesídendõk es
% a szakdolgozat /diplomamunka szo megfeleloen kivalasztando.


%A nyilatkozat szövege TITKOSNAK NEM MINÕSÍTETT dolgozatban a következõ:
%A pontokkal jelölt szövegrészek értelemszerûen a szövegszerkesztõben és
%nem kézzel helyettesítendõk:

\noindent
Alulírott \makebox[4cm]{\dotfill} szakos hallgató, kijelentem, hogy a dolgozatomat a Szegedi Tudományegyetem, Informatikai Intézet \makebox[4cm]{\dotfill} Tanszékén készítettem, \makebox[4cm]{\dotfill} diploma megszerzése érdekében.

Kijelentem, hogy a dolgozatot más szakon korábban nem védtem meg, saját munkám eredménye, és csak a hivatkozott forrásokat (szakirodalom, eszközök, stb.) használtam fel.

Tudomásul veszem, hogy szakdolgozatomat / diplomamunkámat a Szegedi Tudományegyetem Informatikai Intézet könyvtárában, a helyben olvasható könyvek között helyezik el.

\vspace*{4cm}

\begin{tabular}{lc}
Szeged, \today\
\hspace{2cm} & \makebox[6cm]{\dotfill} \\
& aláírás \\
\end{tabular}


\vspace*{2cm}

%A nyilatkozat szövege TITKOSNAK MINÕSÍTETT dolgozatban a következõ:

%\noindent
%Alulírott \makebox[4cm]{\dotfill} szakos hallgató, kijelentem, hogy a dolgozatomat a Szegedi Tudományegyetem, Informatikai Intézet \makebox[4cm]{\dotfill} Tanszékén készítettem, \makebox[4cm]{\dotfill} diploma megszerzése érdekében.

%Kijelentem, hogy a dolgozatot más szakon korábban nem védtem meg, saját munkám eredménye, és csak a hivatkozott forrásokat (szakirodalom, eszközök, stb.) használtam fel.

%Tudomásul veszem, hogy szakdolgozatomat / diplomamunkámat a TVSZ 4. sz. mellékletében leírtak szerint kezelik.

%\vspace*{1cm}

%\begin{tabular}{lc}
%Szeged, \today\
%\hspace{2cm} & \makebox[6cm]{\dotfill} \\
%& aláírás \\
%\end{tabular}

\chapter*{Köszönetnyilvánítás}
\addcontentsline{toc}{section}{Köszönetnyilvánítás}

Ezúton szeretnék köszönetet mondani \textbf{X. Y-nak} ezért és ezért \ldots




%Irodalomjegyzek
\bibliography{mybib}{}
\bibliographystyle{plain}



\end{document}