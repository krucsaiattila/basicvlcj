\chapter*{Tartalmi összefoglaló}
\addcontentsline{toc}{section}{Tartalmi összefoglaló}
\spacing{1.5}

Jelen dolgozat célja egy a nyelvtanulást elősegítő alkalmazás munkamenetének dokumentálása. A feladat egy olyan asztali alkalmazás elkészítése volt, amely képes videofájlok lejátszására, feliratok dinamikus megjelenítésére, valamint a felirat szavaira kattintva azokról különböző fordításokat tud ismertetni a felhasználóval, olyan módon, hogy a lejátszott videó folytonossága nem szakad meg. Ezek után az alkalmazás lehetőséget biztosít egy tudásfelmérő teszt kitöltésére, illetve egy szószedet generálására, amely a felhasználó számára ismeretlen szavakból áll. 

Az szoftver koncepciója egy általános tényre épít, miszerint mindenki szeret filmeket nézni. Mivel rengeteg műfaj áll ezekből a felhasználó rendelkezésére, így mindenki megtalálhatja a neki legmegfelelőbb filmkategóriát, amely filmjeit szívesen nézi. Emellett a filmek a nyelvtudás egyik forrása is lehet, mivel ezek rendszerint idegen nyelvűek, amelyeket később szinkronizálnak anyanyelvűre. Ezek alapján  a kellemest összeköthetjük a hasznossal oly módon, hogy filmnézés közben sajátíthatjuk el egy-egy idegennyelv szavait, kifejezéseit, nyelvtani összetételeit, mindezt szórakozással egybekötve.

Az elkészült alkalmazás a fent említett feltételeket teljesíti, kényelmes felhasználói felületet biztosít a használója számára. Fő felhasználási területe tehát a nyelvtanulás, mivel idegennyelvű filmek nézése közben a felhasználó képes számára ismeretlen szavak jelentéseinek megismerésére. A film megnézése után pedig különböző eszközök állnak rendelkezésre, amellyel a néző a tudását ellenőrizheti, bővítheti.